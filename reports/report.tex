\documentclass[12pt,]{article}
\usepackage[left=1in,top=1in,right=1in,bottom=1in]{geometry}
\newcommand*{\authorfont}{\fontfamily{phv}\selectfont}
\usepackage[]{libertine}


  \usepackage[T1]{fontenc}
  \usepackage[utf8]{inputenc}




\usepackage{abstract}
\renewcommand{\abstractname}{}    % clear the title
\renewcommand{\absnamepos}{empty} % originally center

\renewenvironment{abstract}
 {{%
    \setlength{\leftmargin}{0mm}
    \setlength{\rightmargin}{\leftmargin}%
  }%
  \relax}
 {\endlist}

\makeatletter
\def\@maketitle{%
  \newpage
%  \null
%  \vskip 2em%
%  \begin{center}%
  \let \footnote \thanks
    {\fontsize{18}{20}\selectfont\raggedright  \setlength{\parindent}{0pt} \@title \par}%
}
%\fi
\makeatother




\setcounter{secnumdepth}{0}

\usepackage{color}
\usepackage{fancyvrb}
\newcommand{\VerbBar}{|}
\newcommand{\VERB}{\Verb[commandchars=\\\{\}]}
\DefineVerbatimEnvironment{Highlighting}{Verbatim}{commandchars=\\\{\}}
% Add ',fontsize=\small' for more characters per line
\usepackage{framed}
\definecolor{shadecolor}{RGB}{248,248,248}
\newenvironment{Shaded}{\begin{snugshade}}{\end{snugshade}}
\newcommand{\AlertTok}[1]{\textcolor[rgb]{0.94,0.16,0.16}{#1}}
\newcommand{\AnnotationTok}[1]{\textcolor[rgb]{0.56,0.35,0.01}{\textbf{\textit{#1}}}}
\newcommand{\AttributeTok}[1]{\textcolor[rgb]{0.77,0.63,0.00}{#1}}
\newcommand{\BaseNTok}[1]{\textcolor[rgb]{0.00,0.00,0.81}{#1}}
\newcommand{\BuiltInTok}[1]{#1}
\newcommand{\CharTok}[1]{\textcolor[rgb]{0.31,0.60,0.02}{#1}}
\newcommand{\CommentTok}[1]{\textcolor[rgb]{0.56,0.35,0.01}{\textit{#1}}}
\newcommand{\CommentVarTok}[1]{\textcolor[rgb]{0.56,0.35,0.01}{\textbf{\textit{#1}}}}
\newcommand{\ConstantTok}[1]{\textcolor[rgb]{0.00,0.00,0.00}{#1}}
\newcommand{\ControlFlowTok}[1]{\textcolor[rgb]{0.13,0.29,0.53}{\textbf{#1}}}
\newcommand{\DataTypeTok}[1]{\textcolor[rgb]{0.13,0.29,0.53}{#1}}
\newcommand{\DecValTok}[1]{\textcolor[rgb]{0.00,0.00,0.81}{#1}}
\newcommand{\DocumentationTok}[1]{\textcolor[rgb]{0.56,0.35,0.01}{\textbf{\textit{#1}}}}
\newcommand{\ErrorTok}[1]{\textcolor[rgb]{0.64,0.00,0.00}{\textbf{#1}}}
\newcommand{\ExtensionTok}[1]{#1}
\newcommand{\FloatTok}[1]{\textcolor[rgb]{0.00,0.00,0.81}{#1}}
\newcommand{\FunctionTok}[1]{\textcolor[rgb]{0.00,0.00,0.00}{#1}}
\newcommand{\ImportTok}[1]{#1}
\newcommand{\InformationTok}[1]{\textcolor[rgb]{0.56,0.35,0.01}{\textbf{\textit{#1}}}}
\newcommand{\KeywordTok}[1]{\textcolor[rgb]{0.13,0.29,0.53}{\textbf{#1}}}
\newcommand{\NormalTok}[1]{#1}
\newcommand{\OperatorTok}[1]{\textcolor[rgb]{0.81,0.36,0.00}{\textbf{#1}}}
\newcommand{\OtherTok}[1]{\textcolor[rgb]{0.56,0.35,0.01}{#1}}
\newcommand{\PreprocessorTok}[1]{\textcolor[rgb]{0.56,0.35,0.01}{\textit{#1}}}
\newcommand{\RegionMarkerTok}[1]{#1}
\newcommand{\SpecialCharTok}[1]{\textcolor[rgb]{0.00,0.00,0.00}{#1}}
\newcommand{\SpecialStringTok}[1]{\textcolor[rgb]{0.31,0.60,0.02}{#1}}
\newcommand{\StringTok}[1]{\textcolor[rgb]{0.31,0.60,0.02}{#1}}
\newcommand{\VariableTok}[1]{\textcolor[rgb]{0.00,0.00,0.00}{#1}}
\newcommand{\VerbatimStringTok}[1]{\textcolor[rgb]{0.31,0.60,0.02}{#1}}
\newcommand{\WarningTok}[1]{\textcolor[rgb]{0.56,0.35,0.01}{\textbf{\textit{#1}}}}



\title{CSC8631 Coursework Assignment  }
 



\author{\Large Mariela Ayu Prasetyo
(210407835)\vspace{0.05in} \newline\normalsize\emph{Newcastle
University}  }


\date{}

\usepackage{titlesec}

\titleformat*{\section}{\normalsize\bfseries}
\titleformat*{\subsection}{\normalsize\itshape}
\titleformat*{\subsubsection}{\normalsize\itshape}
\titleformat*{\paragraph}{\normalsize\itshape}
\titleformat*{\subparagraph}{\normalsize\itshape}


\usepackage{natbib}
\bibliographystyle{apsr}
\usepackage[strings]{underscore} % protect underscores in most circumstances



\newtheorem{hypothesis}{Hypothesis}
\usepackage{setspace}


% set default figure placement to htbp
\makeatletter
\def\fps@figure{htbp}
\makeatother

\usepackage{hyperref}
\usepackage{array}
\usepackage{caption}
\usepackage{graphicx}
\usepackage{siunitx}
\usepackage{multirow}
\usepackage{hhline}
\usepackage{calc}
\usepackage{tabularx}
\usepackage{fontawesome}
\usepackage[para,online,flushleft]{threeparttable}

% move the hyperref stuff down here, after header-includes, to allow for - \usepackage{hyperref}

\makeatletter
\@ifpackageloaded{hyperref}{}{%
\ifxetex
  \PassOptionsToPackage{hyphens}{url}\usepackage[setpagesize=false, % page size defined by xetex
              unicode=false, % unicode breaks when used with xetex
              xetex]{hyperref}
\else
  \PassOptionsToPackage{hyphens}{url}\usepackage[draft,unicode=true]{hyperref}
\fi
}

\@ifpackageloaded{color}{
    \PassOptionsToPackage{usenames,dvipsnames}{color}
}{%
    \usepackage[usenames,dvipsnames]{color}
}
\makeatother
\hypersetup{breaklinks=true,
            bookmarks=true,
            pdfauthor={Mariela Ayu Prasetyo (210407835) (Newcastle
University)},
             pdfkeywords = {},  
            pdftitle={CSC8631 Coursework Assignment},
            colorlinks=true,
            citecolor=blue,
            urlcolor=blue,
            linkcolor=magenta,
            pdfborder={0 0 0}}
\urlstyle{same}  % don't use monospace font for urls

% Add an option for endnotes. -----


% add tightlist ----------
\providecommand{\tightlist}{%
\setlength{\itemsep}{0pt}\setlength{\parskip}{0pt}}

% add some other packages ----------

% \usepackage{multicol}
% This should regulate where figures float
% See: https://tex.stackexchange.com/questions/2275/keeping-tables-figures-close-to-where-they-are-mentioned
\usepackage[section]{placeins}


\begin{document}
	
% \pagenumbering{arabic}% resets `page` counter to 1 
%    

% \maketitle

{% \usefont{T1}{pnc}{m}{n}
\setlength{\parindent}{0pt}
\thispagestyle{plain}
{\fontsize{18}{20}\selectfont\raggedright 
\maketitle  % title \par  

}

{
   \vskip 13.5pt\relax \normalsize\fontsize{11}{12} 
\textbf{\authorfont Mariela Ayu Prasetyo
(210407835)} \hskip 15pt \emph{\small Newcastle University}   

}

}






\vskip -8.5pt


 % removetitleabstract

\noindent  

\hypertarget{introduction}{%
\section{Introduction}\label{introduction}}

This report aims to discuss and report any findings to the CSC8631
coursework assignment regarding exploratory data analysis in learning
analytics. Regarding the data set, the students were provided with one
from FutureLearn MOOC. We are then asked to provide any valuable or
non-valuable insights while following the best-practice development
explained thoroughly via the teaching resources available. To ensure we
are following the data-driven process throughout the project, we will
also be adhering to the CRISP-DM methodology. Lastly, we will wrap up
with a conclusion regarding the overall findings.

\hypertarget{business-understanding}{%
\section{Business Understanding}\label{business-understanding}}

The first step of CRISP-DM is the \emph{Business Understanding} step,
where we try to understand what the business wants to solve. In this
particular project, we are given the data set regarding a course in the
FutureLearn MOOC platform. Just like a real-life face-to-face class, we
want to find out whether students are involved in the class or not and
how many students continue to participate until the end of the semester.
We are also interested in finding out how many students leave the course
and, naturally, their reason for doing so. Next, we want to find out
which topics students are most interested to learn about. Lastly, we
want to know the students' opinion of the course and what to improve in
the future. After formulating the previous problems into a sentence, we
come up with the following precise questions :

\begin{itemize}
\tightlist
\item
  Is the students in the course highly engaged? (CHECK): and Are there
  any variables closely related to the participation or engagement rate?
\item
  How many students and what causes students to leave the course?
\item
  What topic interests students the most?
\item
  How are the students who have taken the course feedback?
\end{itemize}

\hypertarget{data-understanding-and-data-preparation}{%
\section{Data Understanding and Data
Preparation}\label{data-understanding-and-data-preparation}}

After formulating the questions, we move on to the \emph{Data
Understanding} and \emph{Data Preparation} part, where we focus on
understanding and formatting the data that assist the business tasks
defined in \emph{Business Understanding}. This phase will consist of:

\begin{itemize}
\item
  Describe data: In this part, we are trying to understand and describe
  the data in a short description. We can do this by examining the data
  format, the number of rows and columns, and the features that are
  accessible.
\item
  Exploring the data: In this section, we are trying to analyse the
  relationship between data and visualise the data. The conclusion and
  visualisation of the data exploration should support and verify the
  business question defined previously. We will tailor the data by
  selecting, cleaning, integrating, and formatting it.
\end{itemize}

\hypertarget{describe-data}{%
\subsection{Describe Data}\label{describe-data}}

Firstly, we are given the data set regarding an online course from the
FutureLearn MOOC platform. The data set consists of several files for
each run from run 1 to run 7. Each run may consist of the following data
in the .csv form (the number inside the bracket denotes how many
variables are there in the file):

\begin{itemize}
\tightlist
\item
  archetype survey response (4)
\item
  enrolments (13)
\item
  leaving survey response (8)
\item
  question response (10)
\item
  step activity (6)
\item
  team members (5)
\item
  video stats (28)
\item
  weekly sentiment survey response (4)
\end{itemize}

\noindent Each of the files consists of different rows (entries), and
all the columns (variables) are stored in a chr format.

\hypertarget{exploring-the-data}{%
\subsection{Exploring the Data}\label{exploring-the-data}}

In this section, we will begin to explore the given data set. In
particular, we want to explore the area where the solution will support
the problem defined in the \emph{Business Understanding} part. There are
three questions and we will explore them one by one.

\hypertarget{question-1}{%
\subsubsection{Question 1}\label{question-1}}

\textbf{What is the participation rate of the students? (CHECK): and Are
there any variables closely related to this?}\\
\hfill\break For the first question, we want to analyse whether students
are highly engaged in the course. There are many ways to check this, but
we will check the full participation rate. We will focus on how many
percentages of students fully participated and finished the material of
the classes. We will also check the duration of the completion for each
student and we can do this by checking the enrolments.csv provided in
the data set.\\
\hfill\break Firstly, we will do some data pre-processing for part 1

\begin{Shaded}
\begin{Highlighting}[]
\CommentTok{\#read file enrolments run 1 to 7}
\NormalTok{files }\OtherTok{=} \FunctionTok{list.files}\NormalTok{(}\AttributeTok{path =} \StringTok{"data/"}\NormalTok{, }
                   \AttributeTok{pattern=}\StringTok{"*enrolments.csv"}\NormalTok{, }\AttributeTok{full.names =}\NormalTok{ T)}

\CommentTok{\#store each run in a single variable}
\ControlFlowTok{for}\NormalTok{ (i }\ControlFlowTok{in} \DecValTok{1}\SpecialCharTok{:}\FunctionTok{length}\NormalTok{(files)) \{}
\NormalTok{  temp }\OtherTok{\textless{}{-}} \FunctionTok{paste}\NormalTok{(}\StringTok{"enrolments"}\NormalTok{, i, }\AttributeTok{sep =} \StringTok{""}\NormalTok{)}
  \FunctionTok{assign}\NormalTok{(temp, }\FunctionTok{read.csv}\NormalTok{(files[i]))}
\NormalTok{\}}
\end{Highlighting}
\end{Shaded}

We read and store each run of enrolments in a single variable for a
later use. After that, we begin to do some processing in our data. We
begin by subsetting the students who fully participated in the course
(there are entries for the fully\_participated columns in the csv data).

\begin{Shaded}
\begin{Highlighting}[]
\NormalTok{enrolments }\OtherTok{\textless{}{-}} 
\NormalTok{  enrolments[}\SpecialCharTok{!}\NormalTok{enrolments}\SpecialCharTok{$}\NormalTok{fully\_participated\_at }\SpecialCharTok{==} \StringTok{""}\NormalTok{ , ]}
\end{Highlighting}
\end{Shaded}

Then, we convert the enrolled\_at and fully\_participated variables of
each run from string format to date format to calculate the duration
later using difftime function. We convert it by using the as.Date
function. It is also worth mentioning that we store the code inside a
function for a more effective writing process for the converting and
calculating duration part.

\begin{Shaded}
\begin{Highlighting}[]
\FunctionTok{as.Date}\NormalTok{(}\FunctionTok{as.character}\NormalTok{(enrolments\_na}\SpecialCharTok{$}\NormalTok{enrolled\_at), }
        \AttributeTok{format =} \StringTok{"\%Y{-}\%m{-}\%d"}\NormalTok{) }
\FunctionTok{as.Date}\NormalTok{(}\FunctionTok{as.character}\NormalTok{(enrolments\_na}\SpecialCharTok{$}\NormalTok{fully\_participated\_at), }
        \AttributeTok{format =} \StringTok{"\%Y{-}\%m{-}\%d"}\NormalTok{)}
\FunctionTok{difftime}\NormalTok{(enrolments\_na}\SpecialCharTok{$}\NormalTok{fully\_participated\_at, }
\NormalTok{         enrolments\_na}\SpecialCharTok{$}\NormalTok{enrolled\_at,}
         \AttributeTok{units =} \FunctionTok{c}\NormalTok{(}\StringTok{"days"}\NormalTok{))}
\end{Highlighting}
\end{Shaded}

After converting it, we calculate the duration between the
fully\_participated\_at and enrolled\_at and store it in a new variable.
We also remove entries or outliers for each run where the period exceeds
365 days or one year.

\begin{Shaded}
\begin{Highlighting}[]
\NormalTok{enrolments1\_na }\OtherTok{\textless{}{-}}\NormalTok{ enrolments1\_na[}\SpecialCharTok{!}\NormalTok{(enrolments1\_na}\SpecialCharTok{$}\NormalTok{duration }
                                   \SpecialCharTok{\textgreater{}} \DecValTok{365}\NormalTok{) , ]}
\end{Highlighting}
\end{Shaded}

Lastly, we calculate the percentage of students who fully participated
in the course proportion to the overall enrollment for each run.

\begin{Shaded}
\begin{Highlighting}[]
\NormalTok{enrolments\_completion\_rate[}\DecValTok{1}\NormalTok{] }\OtherTok{=} 
  \FunctionTok{dim}\NormalTok{(enrolments1\_na)[}\DecValTok{1}\NormalTok{]}\SpecialCharTok{/}\FunctionTok{dim}\NormalTok{(enrolments1)[}\DecValTok{1}\NormalTok{] }\SpecialCharTok{*} \DecValTok{100}
\end{Highlighting}
\end{Shaded}

After preparing all of the data, we can now do the graphing. For the
first graph, we will graph using ggplot2 the duration of completion
against the starting date for each student in each run.

\begin{Shaded}
\begin{Highlighting}[]
\CommentTok{\# Insert plot 1}
\NormalTok{graph1}
\end{Highlighting}
\end{Shaded}

\begin{center}\includegraphics{report_files/figure-latex/unnamed-chunk-7-1} \end{center}

For the second graph, we will include the percentage of students who
fully participated in the course proportion to the overall enrollment
for each run.

\begin{Shaded}
\begin{Highlighting}[]
\CommentTok{\# Insert plot 2}
\NormalTok{graph2}
\end{Highlighting}
\end{Shaded}

\begin{center}\includegraphics{report_files/figure-latex/unnamed-chunk-8-1} \end{center}

\hypertarget{discussion}{%
\subsubsection{Discussion}\label{discussion}}

In graph 1, we can see that duration of students completing the courses
varies but is primarily concentrated in 0 to 150 days. We can also see
that the number of people who fully participated in the course decreases
for each run. This means that the class gets less engaging as time goes
by, and different approaches are needed to boost the engagement rate of
the students. Additionally, the graph shows runs that open in the latter
half of the year gain more enrollment than those that begin in the first
half of the year. This pattern could be taken into consideration when
opening a new run in the future.\\
\hfill\break Graph 2 shows the percentage of students who fully
participated in proportion to the overall enrolments against each run.
We can see that it peaked at 12.5\% for run one and stays under 5\%
after that. It confirms the statement before that the students who fully
participated decreased for each run.

\hypertarget{question-2}{%
\subsubsection{Question 2}\label{question-2}}

\textbf{How many students and what causes students to leave the
course?}\\
\hfill\break For the second question, we want to analyse how many
students leave the course, what time they quit, and why. Same as before,
we will begin by pre-processing the data from leaving survey responses.
It is worth mentioning that we assume people who formally quit the
course are all required to fill the survey, so the number of students
who formally leave the class is equal to the number of the survey
entries.

\begin{Shaded}
\begin{Highlighting}[]
\CommentTok{\#read file archetype survey responses run 1 to 7 }
\CommentTok{\#and store it in a variable}
\NormalTok{files }\OtherTok{=} \FunctionTok{list.files}\NormalTok{(}\AttributeTok{path =} \StringTok{"data/"}\NormalTok{, }
                   \AttributeTok{pattern=}\StringTok{"*leaving{-}survey{-}responses.csv"}
\NormalTok{                   , }\AttributeTok{full.names =}\NormalTok{ T)}

\ControlFlowTok{for}\NormalTok{ (i }\ControlFlowTok{in} \DecValTok{1}\SpecialCharTok{:}\FunctionTok{length}\NormalTok{(files)) \{}
\NormalTok{  temp }\OtherTok{\textless{}{-}} \FunctionTok{paste}\NormalTok{(}\StringTok{"leaving"}\NormalTok{, i, }\AttributeTok{sep =} \StringTok{""}\NormalTok{)}
  \FunctionTok{assign}\NormalTok{(temp, }\FunctionTok{read.csv}\NormalTok{(files[i]))}
\NormalTok{\}}
\end{Highlighting}
\end{Shaded}

We read the file and store each run data in a variable similar to
before. We can then move on to the processing part. We begin by merging
the enrolment and leaving data to extract the enrolment\_at column. The
column later will be used to calculate the duration between the starting
and leaving date.

\begin{Shaded}
\begin{Highlighting}[]
\NormalTok{leaving4 }\OtherTok{\textless{}{-}} \FunctionTok{merge}\NormalTok{(leaving4, enrolments4, }\AttributeTok{by =} \StringTok{"learner\_id"}\NormalTok{)}
\end{Highlighting}
\end{Shaded}

Then, we drop the columns unrelated to our observations, such as id,
last\_completed\_step\_at, etc.

\begin{Shaded}
\begin{Highlighting}[]
\NormalTok{leaving4 }\OtherTok{\textless{}{-}}\NormalTok{ leaving4[}\SpecialCharTok{{-}}\FunctionTok{c}\NormalTok{(}\DecValTok{5}\SpecialCharTok{:}\DecValTok{8}\NormalTok{, }\DecValTok{10}\SpecialCharTok{:}\DecValTok{20}\NormalTok{)]}
\end{Highlighting}
\end{Shaded}

Like before, we convert the date from string format to date format and
calculate the duration between the starting and leaving dates using
difftime function.

\begin{Shaded}
\begin{Highlighting}[]
\FunctionTok{as.Date}\NormalTok{(}\FunctionTok{as.character}\NormalTok{(leaving}\SpecialCharTok{$}\NormalTok{enrolled\_at), }
        \AttributeTok{format =} \StringTok{"\%Y{-}\%m{-}\%d"}\NormalTok{) }
\FunctionTok{as.Date}\NormalTok{(}\FunctionTok{as.character}\NormalTok{(leaving}\SpecialCharTok{$}\NormalTok{left\_at), }
        \AttributeTok{format =} \StringTok{"\%Y{-}\%m{-}\%d"}\NormalTok{)}
\FunctionTok{difftime}\NormalTok{(leaving}\SpecialCharTok{$}\NormalTok{left\_at, leaving}\SpecialCharTok{$}\NormalTok{enrolled\_at, }
         \AttributeTok{units =} \FunctionTok{c}\NormalTok{(}\StringTok{"days"}\NormalTok{))}
\end{Highlighting}
\end{Shaded}

We also compute how many percentages of students formally quit the
course.

\begin{Shaded}
\begin{Highlighting}[]
\NormalTok{leaving\_num[}\DecValTok{1}\NormalTok{] }\OtherTok{=} \FunctionTok{dim}\NormalTok{(leaving4)[}\DecValTok{1}\NormalTok{]}\SpecialCharTok{/}\FunctionTok{dim}\NormalTok{(enrolments4)[}\DecValTok{1}\NormalTok{] }\SpecialCharTok{*} \DecValTok{100}
\end{Highlighting}
\end{Shaded}

Lastly, we bind the data and quantify the students' leaving reason for
easier graphing. We will then calculate the number of times each cause
got picked. After presenting the graph, we will discuss it in the
discussion area for a more detailed explanation.

\begin{Shaded}
\begin{Highlighting}[]
\CommentTok{\#bind the data}
\NormalTok{merged\_leaving }\OtherTok{\textless{}{-}} \FunctionTok{do.call}\NormalTok{(}\StringTok{"rbind"}\NormalTok{, }\FunctionTok{list}\NormalTok{(leaving4, leaving5, }
\NormalTok{                                        leaving6, leaving7))}

\CommentTok{\#quantify leaving reason for easier graphing}
\NormalTok{merged\_leaving }\OtherTok{\textless{}{-}}\NormalTok{ merged\_leaving }\SpecialCharTok{\%\textgreater{}\%}
  \FunctionTok{group\_by}\NormalTok{(leaving\_reason) }\SpecialCharTok{\%\textgreater{}\%}
  \FunctionTok{mutate}\NormalTok{(}\AttributeTok{reason\_num =} \FunctionTok{cur\_group\_id}\NormalTok{())}
\end{Highlighting}
\end{Shaded}

After the data preparation completes, we will begin to plot using the
library of ggplot2. The third graph will demonstrate the duration from
starting date until leaving date against the leaving date.

\begin{Shaded}
\begin{Highlighting}[]
\CommentTok{\# Insert plot 3}
\NormalTok{graph3}
\end{Highlighting}
\end{Shaded}

\begin{center}\includegraphics{report_files/figure-latex/unnamed-chunk-15-1} \end{center}

For the fourth graph, we will include the percentage of students who
leave in the course in proportion to the overall enrollment for each
run.

\begin{Shaded}
\begin{Highlighting}[]
\CommentTok{\# Insert plot 4}
\NormalTok{graph4}
\end{Highlighting}
\end{Shaded}

\begin{center}\includegraphics{report_files/figure-latex/unnamed-chunk-16-1} \end{center}

The last graph will present the bar plot consisting number of entries
for each reason.

\begin{Shaded}
\begin{Highlighting}[]
\CommentTok{\# Insert plot 5}
\NormalTok{graph5}
\end{Highlighting}
\end{Shaded}

\begin{center}\includegraphics{report_files/figure-latex/unnamed-chunk-17-1} \end{center}

\hypertarget{discussion-1}{%
\subsubsection{Discussion}\label{discussion-1}}

In graph 3, we can see no distinct` patterns regarding either the
leaving date or the duration from starting to leaving. Students leaving
times are spread out evenly, just like the duration of quitting. Thus,
we can not extract any valuable insight from this graph. For graph 4,
the leaving rate of students (in proportion to the overall enrollments
number) ranges from 2\%-5\% and is also relatively low. Comparing graphs
2 and 4, the data suggest that many students dropped the class and did
not leave the course formally.\\
\hfill\break Graph five shows the bar plot consisting of the number of
entries for each reason. Each number represents the reasons below

\begin{itemize}
\tightlist
\item
  I don't have enough time (1)
\item
  I prefer not to say (2)
\item
  Other (3)
\item
  The course required more time than I realised (4)
\item
  The course was too easy (5)
\item
  The course was too hard (6)
\item
  The course wasn't what I expected (7)
\item
  The course won't help me reach my goals (8)
\end{itemize}

We can see from the graph that the biggest reason for students leaving
the course is reason one, which is ``I don't have enough time.'' It then
followed by reason three, ``Other'' and the other reason fairs about the
same. The high count of reason one suggests that many students did not
have enough time to learn and complete the course material. This
interpretation could be considered when designing the structure of
future course runs. The teaching and quiz interval could be made more
sparse, thus allowing more time flexibility for the students. Since many
students answered ``Other,'' it indicates that their reason for leaving
is not stated in the options above. Thus, to enable a more directed
feedback process in the future, more options should be listed, or
students should have the chance to write their reason for leaving
directly.

\hypertarget{question-3}{%
\subsubsection{Question 3}\label{question-3}}

\textbf{What topic interests students the most?}\\
\hfill\break For the third question, we will see which topic interests
students the most through the video viewing status provided in the
video-stats.csv. In particular, we will check variable
viewed\_onehundred\_percent and see which topic have the highest rating.
We will begin by pre-processing the data first.

\hypertarget{additional-notes}{%
\subsubsection{Additional notes}\label{additional-notes}}

Configuration setting

\hypertarget{conclusion}{%
\section{Conclusion}\label{conclusion}}

AAA





\newpage
\singlespacing 
\bibliography{master.bib}

\end{document}